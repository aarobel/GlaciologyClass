\documentclass[12pt]{article}
\usepackage[utf8]{inputenc}	% Para caracteres en español
\usepackage{amsmath,amsthm,amsfonts,amssymb,amscd}
\usepackage{multirow,booktabs}
\usepackage[table]{xcolor}
\usepackage{fullpage}
\usepackage{lastpage}
\usepackage{enumitem}
\usepackage{fancyhdr}
\usepackage{mathrsfs}
\usepackage{wrapfig}
\usepackage{setspace}
\usepackage{calc}
\usepackage{natbib}
\usepackage{multicol}
\usepackage{cancel}
\usepackage[retainorgcmds]{IEEEtrantools}
\usepackage[margin=3cm]{geometry}
\usepackage{amsmath}
\newlength{\tabcont}
\setlength{\parindent}{0.0in}
\setlength{\parskip}{0.05in}
\usepackage{empheq}
\usepackage{framed}
\usepackage[most]{tcolorbox}
\usepackage{xcolor}
\colorlet{shadecolor}{orange!15}
\parindent 0in
\parskip 12pt
\geometry{margin=1in, headsep=0.25in}
\theoremstyle{definition}
\newcommand{\pd}[2]{\frac{\partial {#1}}{\partial {#2}}}
\newcommand{\mytilde}{\raise.17ex\hbox{$\scriptstyle\mathtt{\sim}$}}
\newtheorem{defn}{Definition}
\newtheorem{reg}{Rule}
\newtheorem{exer}{Exercise}
\newtheorem{note}{Note}
\begin{document}

\thispagestyle{empty}

\begin{center}
{\LARGE \bf Ice Stream Velocity}\\
{\large GT EAS 4803/8803}\\
\end{center}

Let us consider an ice stream of half-width $W$, which is primarily flowing in one direction.

\begin{figure}[h]
  \begin{center}
\includegraphics[width=0.3\textwidth]{ISvelfig.pdf}
  \end{center}
  \vspace{-20pt}
\end{figure}

First, we can go back to the high-order ice flow approximation for ice flow along the x-direction:
\begin{equation}
4 \pd{}{x} \left( \eta \pd{u}{x} \right) + 2 \pd{}{x} \left( \eta \pd{v}{y} \right) + \pd{}{y} \left( \frac{\eta}{2} \left(\pd{u}{y} + \pd{v}{x} \right) \right) + \pd{}{z} \left(\eta \pd{u}{z} \right) = \rho g \pd{h}{x}
\end{equation}
Let us make two key assumptions: (1) flow is primarily along the x-direction ($u>>v$) and the changes in velocity along flow are negligible ($\pd{u}{x}$ is small). This leaves
\begin{equation}
\pd{}{y} \left( \frac{\eta}{2} \pd{u}{y} \right) + \pd{}{z} \left(\eta \pd{u}{z} \right) = \rho g \pd{h}{x}
\end{equation}
Reminding ourselves that $\eta \pd{u}{z} = \tau_{xz}$ and $\tau_d = \rho g \pd{h}{x}$. We can then integrate this equation with respect to depth ($z$), producing
\begin{equation}
h \pd{}{y} \left( \frac{\eta}{2} \pd{u}{y} \right) + \left(\tau_{xz}|_{z=h} - \tau_{xz}|_{z=0}  \right) = \tau_d
\end{equation}
Shear stress at the ice sheet surface should be negligible ($\tau_{xz}|_{z=h} = 0$), and $\tau_{xz}|_{z=0}$ is simply the basal shear stress, $\tau_b$, leaving
\begin{equation}
h \pd{}{y} \left( \frac{\eta}{2} \pd{u}{y} \right) = \tau_d - \tau_b
\end{equation}
We can now integrate along the y-direction, starting from the center of the ice stream where $\tau_{xy}$ will be small to the ice stream edge (i.e. the ``shear margin'')
\begin{equation}
h \int_0^y \pd{}{y} \left( \frac{\eta}{2} \pd{u}{y} \right) dy = \int_0^y \left( \tau_d - \tau_b \right) dy
\end{equation}
Giving
\begin{equation}
 \frac{\eta}{2} \pd{u}{y} = \frac{y}{h} \left( \tau_d - \tau_b \right)
\end{equation}
It is now useful to re-examine the effective viscosity of ice in the higher-order approximation
\begin{equation}
\eta = A^{-1/n} \left(\frac{1}{2} \left(\dot{\epsilon}_{xx}^2 + \dot{\epsilon}_{yy}^2 + \dot{\epsilon}_{zz}^2 \right) + \dot{\epsilon}_{xz}^2 + \dot{\epsilon}_{xy}^2 \right)^{\frac{1-n}{2n}}
\end{equation}
Vertical strain rates and along-flow strain rates are all negligible compared to the cross-flow strain rates, causing all these terms to drop out except the last one
\begin{equation}
\eta = A^{-1/n} \left(\frac{1}{2} \dot{\epsilon}_{xy} \right)^{\frac{1-n}{n}}
\end{equation}
Inserting this back into the momentum balance above (noting that $\dot{\epsilon}_{xy} = \pd{u}{y}$), we have
\begin{equation}
 \frac{1}{2} \left( \pd{u}{y} \right)^{1/n} = \frac{y}{h} \left( \tau_d - \tau_b \right)
\end{equation}
This equation can be integrated one final time in the y-direction
\begin{equation}
\int_0^y \pd{u}{y} dy = \int_0^y \frac{2^n A y^n}{h^n} \left( \tau_d - \tau_b \right)^n dy
\end{equation}
Leading to (with some reorganization)
\begin{shaded}
\begin{equation}
u(y) = U_d \left[1-\left( \frac{y}{W} \right)^{n+1} \right] W^{n+1} \left(\tau_d - \tau_b \right)^n
\end{equation}
where $U_d = \frac{2^n A}{h^n (n+1)}$.
\end{shaded}
A plot of the shape of velocity in an ice stream profile can be seen below.

\begin{figure}[h]
  \begin{center}
\includegraphics[width=0.4\textwidth]{uy.eps}
  \end{center}
  \vspace{-20pt}
\end{figure}

\bibliography{/Users/robel/Dropbox/Docs/refs.bib}
\bibliographystyle{apalike}

\end{document}